\documentclass[12pt, letterpaper]{article}

\newcommand{\acronym}[1]{{\small{#1}}}
\newcommand{\project}[1]{\textsl{#1}}
\newcommand{\apogee}{\project{\acronym{APOGEE}}}

\newcommand{\unit}[1]{\mathrm{#1}}
\newcommand{\m}{\unit{m}}
\newcommand{\s}{\unit{s}}
\newcommand{\mps}{\m\,\s^{-1}}

\newcommand{\dd}{\mathrm{d}}
\newcommand{\given}{\,|\,}
\newcommand{\normal}{{\mathcal{N}}}

\setlength{\parindent}{1.0\baselineskip}
\linespread{1.09}
\raggedbottom
\sloppy
\sloppypar
\frenchspacing

\begin{document}

\section*{\raggedright The dimensionality of chemical-abundance space, in theory and in fact}

\noindent
by DWH, RB, MKN, HWR, others

\paragraph{Abstract:}
Models of interstellar chemical evolution involving supernovae and AGB stars can produce non-trivial abundance distributions.
However, if the supernovae ejecta and stellar winds are not somehow fractionated, they cannot produce arbitrary diversity:
There are a finite range of nucleosynthetic pathways, and these are combined in finite ways in stars of different masses and abundances.
This remains true even if we consider not just the best-in-class yield tables for supernovae and RGB stars, but consider arbitrary modifications of these (subject to perhaps some smoothness constraints), or even if we permit arbitrary mixtures of supernovae, not constrained to be consistent with any reasonable stellar mass function.
Here we use measurements of 17 elements (excluding C and N, which are affected by age) in red-giant stars, measured by the \apogee\ project, to demonstrate that chemical abundance ratios X/Fe span a space of much higher dimensionality than can be achieved in any reasonable chemical-abundance model.
We take three approaches to measuring the dimensionality of the space.
In the first, we treat 7 open clusters observed in the \apogee\ data as mono-abundance control points in the 17-dimensional chemical space, and look at their locations in the space and the dimensionality of the simplex they form.
In the second, we use an affine-invariant method to look at the excess variance in the distribution of YYY red-giant stars in chemical space over what is expected from measurement noise.
In the third, we deconvolve the observed distribution using the measurement noise variances and the extreme deconvolution algorithm.
We find that the open clusters form a simplex of maximal dimensionality, which indicates that their abundances live in a space that is no less than 6-dimensional, and that the red-giant stars span at least YYY dimensions.
The latter conclusion is somewhat sensitive to what we assume about the abundance measurement precision.
The fact that the \apogee\ experiment does not (yet) have any measurements for r- and s-process elements only strengthens conclusions about the emprirical dimensionality.

\section{Introduction}

Wassup

\paragraph{Acknowledgements:}
Thanks to DFM, Weinberg, Ting.

\end{document}
